\documentclass[a4paper,11pt]{report}
\usepackage[utf8]{inputenc}
\usepackage[T1]{fontenc}
\usepackage[francais]{babel}
\usepackage{graphicx}
\usepackage{geometry}



\title{Machines Virtuelles}
\author{Edgar RODRÍGUEZ}


\begin{document}
\maketitle

\subsection{Machines virtuelles}
La virtualisation consiste à faire fonctionner sur un seul ordinateur plusieurs
systèmes d'exploitation comme s'ils fonctionnaient sur des ordinateurs distincts.
Une machine virtuelle est un conteneur fermement isolé capable d’exécuter ses
propres système d’exploitation et applications, à l’instar d’un ordinateur physique.
Elle se comporte exactement comme un ordinateur physique, elle est totalement indépendante du matériel physique sous-jacent. Par exemple, on peut configurer une machine virtuelle avec des composants virtuels (comme un processeur, une carte réseau) qui diffèrent des composants physiques présents dans la machine hôte ce qui donne de nombreux avantages.
 On
appelle serveur privé virtuel (Virtual Private Server ou VPS) ou encore
environnement virtuel (Virtual Environment ou VE) ces ordinateurs virtuels.
Indépendance vis-à-vis matériel 


Les machines virtuelles constituent un module fondamental d’une solution bien plus conséquente : l'infrastructure virtuelle. Tandis qu’une machine virtuelle reproduit les ressources matérielles d’un ordinateur complet, une infrastructure virtuelle représente l’interconnexion des ressources matérielles d’une infrastructure informatique complète, en y incluant ordinateurs, périphériques réseau et ressources de stockage partagées. Des organisations de toutes tailles utilisent des solutions VMware pour configurer un serveur virtuel et des infrastructures de postes de travail virtuels visant à améliorer la disponibilité, la sécurité et la gérabilité des applications critiques pour l’entreprise. 

\subsection{OpenVZ}
OpenVZ est une technologie base de Virtuozzo développé par SWsoft sous licence conformément aux termes de la GNU GPL version 2, de virtualisation au niveau du système d'exploitation Linux (niveau noyau) ce qui permet une meilleure gestion des ressources et des performances accrues par rapport à une virtualisation système. OpenVZ permet à un serveur physique d'exécuter plusieurs instances isolées du système d'exploitation, appelés environnements virtuels (VE).
Par rapport aux machines virtuelles telles que VMware, VirtualBox et les technologies de virtualisation comme Xen, OpenVZ offre moins de flexibilité dans le choix du système d'exploitation: il faut que le système hôte et le système invité soit de base Linux (bien que les distributions de GNU / Linux peut être différent dans différentes VE). Cependant, la virtualisation OpenVZ niveau de fonctionnement du système offre de meilleures performances, l'évolutivité, la densité, la gestion dynamique des ressources, et la facilité d'administration que les solutions de rechange.
OpenVZ est une base de Virtuozzo est un logiciel commercial développé par SWsoft, Inc OpenVZ est un produit de logiciel libre sous licence conformément aux termes de la GNU GPL version 2. 
\begin{itemize}
  \item 
\end{itemize}

\subsection{KVM}
KVM permet la virtualisation de tout système d'exploitation sur des processeurs d'architectures x86 disposant des technologies Intel VT ou AMD-V.
KVM a lieu en tant que module du noyau Linux, ce module exporte un périphérique appelé / dev / kvm permet un mode d'hôtes en plus d'autres modes d'utilisation traditionnels qui prend en charge le noyau. Dans un arbre traditionnel / dev dispositifs sont communs à tous les processus dans l'espace utilisateur, mais dans / dev / kvm à chaque processus qui est lancé est associé à un mappage de périphérique différent.
Cela garantit une isolation entre les machines virtuelles. Lorsque le module est installé KVM, le noyau Linux devient un hyperviseur. Un hyperviseur peut exploiter tous les avantages d'un noyau Linux standard.
Utilisation de systèmes d'exploitation invités KVM peut lancer dans l'espace utilisateur. Chaque système d'exploitation invité est un processus distinct au sein du système d'exploitation hôte. Ci-dessus mais en dessous des systèmes d'exploitation invités il ya une couche très mince (QEMU dans ce cas), qui est l'interface pour la gestion, le contrôle et la configuration des machines virtuelles.

\subsection{Proxmox VE}

Proxmox Virtual Environment est un logiciel libre de virtualisation, plus précisément un hyperviseur de machine virtuelle. Il est développé et maintenu par Proxmox Server Solutions GmbH.
C'est une solution de virtualisation "bare metal" c'est-à-dire qu'on commence à partir d'un serveur vide et qu'il n'y a donc aucune besoin d'installer un système d'exploitation auparavant.
Proxmox VE installe les outils complets du système d'exploitation et de gestion rapidement.
Le logiciel inclut:
\begin{itemize}
  \item Système d'exploitation complet (Debian Lenny 64 bits)
  \item Interface web pour l'administration et la surveillance (création de VM de différents OS, démarrage et arrêt sauvegarde et restore et Migration).
  \item Noyau Proxmox VE avec support d'OpenVZ et de KVM.
  \item Outils de sauvegarde et de restauration.
  \item Partitionnement de disque dur avec LVM2.
  \item Une plateforme de virtualisation (et de 
supervision de VM) pour les opérations de base: Créer, détruire, paramétrer, lancer, arrêter, sauvegarder et déplacer.
  \item Fonctions de clustering qui permet la migration à chaud des machines virtuelles d'un serveur physique à un autre (en utilisant un stockage partagé).
Un «cluster» Proxmox est un regroupement de plusieurs serveurs physiques composé de 1 ou plusieurs noeuds.
\end{itemize}

Système requis:
\begin{itemize}
   \item CPU 64 bits (Intel EM64T ou AMD64)
   \item 2 GB de RAM ou plus (pas limité grâce au noyau 64 bits)
   \item CPU 64 bits (Intel EM64T ou AMD64), microprocesseur multi cœur recommandé
Carte-mère et BIOS compatible Intel VT/AMD-V (pour le support de la virtualisation par KVM)
\end{itemize}

\subsubsection{Installation Proxmox VE}
La configuration la plus important au commence de l'installation.
\begin{figure} [h]
\begin {center}
\includegraphics[width=0.5\textwidth]{img/installation.png}
\caption{Configuration du réseau }
\end {center}
\end{figure}

Proxmox VE met à disposition une interface d’adminsitration web, pour s’y connecter, entrez l’adresse IP de la machine ou une URL y pointant. Une fois loggué on arrive sur la fenêtre suivante :

\begin{figure} [h]
\begin {center}
\includegraphics[width=0.5\textwidth]{img/premiere.png}
\caption{Fenêtre aprés de se logguer. }
\end {center}
\end{figure}


\end{document} 