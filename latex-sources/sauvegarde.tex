\documentclass[a4paper,11pt]{report}
\usepackage[utf8]{inputenc}
\usepackage[T1]{fontenc}
\usepackage[francais]{babel}
\usepackage[top=2cm, bottom=2cm, left=2cm, right=2cm]{geometry}

% Title Page
\title{Sauvegarde}
\author{Clément Ghnassia}


\begin{document}
\maketitle

% \frontmatter

\chapter{Introduction}

\paragraph{}
En Informatique, une sauvegarde consiste à faire des copies de données qui pourront être utilisées après un évènement de pertes de données.

\paragraph{}
La sauvegarde a deux buts :
\begin{itemize}
  \item La restauration de données après que celles ci aient été perdues ou endomagées. En effet, la perte de données est courant pour les utilisateurs. 67\% d'entre eux ont déja eu ce type d'experience.
  \item La restauration de données plus anciennes. Ici, la sauvegarde peut s'apparenter à de l'archivage.
\end{itemize}

\paragraph{}
Le fait d'avoir une bonne stratégie de sauvegarde n'est qu'un élément parmi tant d'autres dans un plan de restauration après un desastre. En effet les administrateurs ont tendance à penser que faire des sauvegardes régulières est suffisant : il n'en est rien.
Il faut penser a plein d'autres situations comme la perte totale du système et des applications, la destruction d'une base de données, ... pour envisager une reprise d'activité rapide.

\paragraph{}
Un système de sauvegarde doit prendre en compte la taille des données qu'il faut sauvegarder. En effet, même si cela dépend tu type de données et de la compression plus ou moins efficace, il faut comprendre qu'une sauvegarde du système contiendra au moins une copie de toutes les données.
Les capacité de stockages qui en découlent sont donc considérables, et c'est donc un facteur majeur dans le choix d'une stratégie de sauvegarde.
Il existe plusieurs modèles pour la sauvegarde de données, chacun étant adapté aux différentes situations, comme la quantité de données à sauvegarder, la capacité de stockage pour les sauvegardes, la redondance des données dans la sauvegarde, la sécurité des données, ou encore la rapidité de la restauration.

\paragraph{}
Avant que les données soit réellements sauvegardées, elles sont selectionnées, extraites et manipulées. Lorsque les données sont importantes, cela peut vite devenir fastidieux.
Des techniques ont été mises aux point pour optimiser ces procédures. Cela peut aller de la résolution de problèmes lors de sauvegardes à chaud, de la compression, du chiffrement ou de la déduplication des données.
Il ne faudra pas non plus négliger les facteurs humains dans les processus de sauvegarde et de restauration, qui tendent tout de même à etre de plus en plus automatisés (pour les sauvegardes du moins).

%\mainmatter

\chapter{Le stockage}

\section{Les modèles d'entrepôts de données}
La première étape dans l'élaboration d'une stratégie de sauvegarde est de trouver le ou les modèles qui correspondront le mieux avec la situation, les contraintes et les objectifs recherchés, 
Le choix d'un modèle est primordial et doit être mûrement réfléchit. Il définira comment les données seront stockées et organisées. Cela aura donc un impact important en terme d'espace disque nécessaire, de sécurité, ou encore de vitesse et de simplicité lors d'une restauration

\subsection{Non structurée}
Même si cela semble la méthode la plus simple à mettre en oeuvre, elle n'est pas très efficace lors d'une perte de données et n'offre pas un très haut niveau de sécurité en terme de restauration de données.
De plus, elle nécessite une organisation méthodique de l'administrateur.
Cette méthode consiste simplement à stocker des sauvegardes avec des informations sur la nature de ces sauvegardes et les dates auquelles elles ont été réalisées.

\subsection{Images du système}
Dans ce type de modèle, il s'agit de stocker une image complète du système à un instant donné. Une des applications courrantes consiste à faire une image afin de copier à l'identique la configuration d'un système sur d'autres machines.
Pas très performant, surtout en ce qui concerne l'optimisation de l'espace disque, il est de moins en moins utilisé pour la sauvegarde.

\subsection{Différentielle}
Cette méthode consiste a réaliser une sauvegarde complète des données souhaitées, puis d'enregistrer à interval régulier toutes les différences depuis la dernière sauvegarde complète.
Cela permet de ne pas dupliquer les données qui n'ont pas changées et offre donc un gain d'espace considérable, dépendant bien evidemment du nombre de données qui peuvent changer d'une sauvegarde à l'autre.
Pour restaurer ces données, on aura besoin de la dernière sauvegarde complète et éventuellement de la dernière sauvegarde avant la date du système souhaitée pour la restauration.

\subsection{Incrémentale}
Cette méthode repose sur le même principe qu'une sauvegarde différentielle, à la différence que lors d'une sauvegarde incrémentale, seul les données modifiées depuis la dernière sauvegarde, complètes ou non, seront enregistrées.
L'avantage par rapport à une sauvegarde différentielle réside dans un gain d'espace disque, car il y'aura moins de données redondantes. En revanche, la restauration s'avèrera plus longue et plus fastidieuse étant donné qu'on aura besoin de rejouer toutes les sauvegardes depuis la dernière sauvegarde complète et éventuellement jusqu'à la dernière sauvegarde incrémentale à la date désirée.

\subsection{Delta inversé}
Ici, le fonctionnement est inversé.
En effet, plutôt que de faire une sauvegarde complète et d'enregistrer les modifications par la suite comme c'est le cas avec les sauvegarde difféntielles et incrémentales, on aura toujours une sauvegarde complète (un mirroir) correspondant à l'état des données à l'instant de la dernière sauvegarde effectuée.
Le fonctionnement est trivial : après avoir effectué une sauvegarde complète, chaque sauvegarde inversée aura pour effet une synchronisation du mirroir avec l'état réel des données. En même temps, les modifications qui permettront de revenir à un état antérieur seront enregistrées.
Ainsi, pour restaurer les données au dernier état, il suffira juste de restaurer le mirroir, et une restauration à un état antérieur impliquera de rejouer les sauvegardes de manière décroissante dans le temps depuis le mirroir jusqu'au moment souhaité.

\subsection{Protection des données en continu (CDP)}
Aussi appelé sauvegarde en temps réel, c'est la méthode la plus sûre en terme de protection des données.
Au lieu d'effectuer des sauvegardes périodiquement, les modifications sont envoyées instantanément vers l'espace de stockage dédié à la sauvegarde à travers le réseau.
Bien que faisant penser à de la réplication (pour de la haute disponibilité par exemple), cela est complètement différent car ce n'est pas une copie parfaite des données qui est réalisée, mais un système de logs et de journaux qui enregistrent les modifications et qui permettront de revenir à un état antérieur dans le cas d'une éventuelle restauration.

%tableau récapitulatif
\begin{table}[h]
 \begin{center}
 \begin{tabular}{|c|c|c|c|c|c|}
  \hline
    Type & Taille & Vitesse de sauvegarde & Vitesse de restauration & Intégrité & Simplicité \\
  \hline
    Non structurée & N & N & O & N & O \\
  \hline
    Image système & N & N & O & N & O \\
  \hline
    Différentielle & O & O & N & O & N \\
  \hline
    Incrémentale & O & O & N & O & N \\
  \hline
    Delta inversée & O & O & O & O & N \\
  \hline
    CDP & O & O & O & O & N \\
  \hline
 \end{tabular}
 \caption{Tableau récapitulatif sur les modèles d'entrepôts de données}
 \end{center}
\end{table}


\section{Dispositifs de stockage}
Aujourd'hui, les périphériques pour le stockage de données sont nombreus et variés. Il s'agira ici d'analyser chacun de ces dispositifs et d'en comprendre les avantages et es inconvénients, en particulier dans le cas d'une utilisation pour la sauvegarde.

\subsection{Bandes magnétiques}
Aujourd'hui encore, c'est sans doute le support le plus utilisé pour le stockage de données, plus particulièrement pour la sauvegarde et l'archivage.
La raison est tout d'abord financière. En effet, même si la tendance est à une réduction des différences de prix entre les bandes magnétiques et les disques durs, c'est encore actuellement la solution la moins chère à capacité de stockage égale.
Il faut noter que les bandes magnétiques sont en adéquation avec les contraintes de sauvegarde car elles fournissent un accès séquentiel, ce qui a un impact bénéfique sur les vitesses de lecture et d'écriture.
De plus, on sait que les bandes mangétiques, si elles sont correctement conservées, assureront l'intégrité de leur contenu sur plusieurs décénies.
Enfin, les bandes magnétiques sont utilisées depuis un certain temps, et leurs caractéristiques sont donc totalement maîtrisées.
En revanche, le problème survient au niveau des formats. En effet, la plupart du temps les formats de bandes mangétiques sont nombreux et souvent propriétaires. Utiliser des bandes magnétiques engendrera donc une dépendance forte vis à vis des constructeurs, et ajoute des contraintes en cas de changement de solution.
De plus, pour les grosses structures, l'investissement sera d'autant plus important que l'achat d'un automate sera nécessaire.

\subsection{Disques durs}
Ce moyen est de plus en plus utilisé pour la sauvegarde et l'archivage, mais dans bien des cas, les solutions à base de bandes mangétiques lui sont toujours préférées, et ce à juste titre.
A première vue, l'utilisation de de disques dur pour la sauvegarde et l'archivage de données semblerait interessante.
En effet, le prix des disques durs se rapproche des bandes magnétiques à capacité équivalente. De plus, les disques durs ont des avantages certains par rapport aus bandes mangétiques en terme de temps d'accès, de disponibilité, de capacité, et de simplicité d'utilisation.
Ils ne nécessitent pas d'équipement spécifique, et permettent une utilisation dans des contextes différents, s'adpatant plus aux besoins, comme par exemple leur emplacement, aussi bien au niveau local que sur le réseau.
On pourra aussi ajouter des procédés permettant de réduire les tailles des sauvegardes, comme des techniques de de-duplication.
Mais l'utilisation de ce type de support dans un contexte de sauvegarde, et surtout d'archivage possède un inconvénient de taille. Tout d'abord la fragilité du support n'est pas adapté a ce type d'utilisation et n'assure donc pas l'intégrité des données stockées. 
En effet là ou la manipulation et l'externalisation des sauvegardes est courante, la fragilité des disques durs est un point critique.
De plus, il n'a pas été établit sur combien de temps les données pourraient être conservées sur le long terme sans risque d'altération de celles-ci.
Si cette contrainte n'est pas insurmontable pour des sauvegardes où la conservation de données sur le long terme en vue d'une restauration n'aurait pas beaucoup de sens, elle l'est en revanche beaucoup plus pour l'archivage, qui nécessite souvent une conservation des données à long terme.

\subsection{Supports optiques}
Si ces supports sont courants et très utilisés par les particuliers pour stocker des données, de par leur prix et leur simplicité d'utilisation, il ne semble pas que ce soit un support adapté dans le cas de sauvegarde et d'archivage de quantités importante de données, comme c'est souvent le cas dans une entreprise.
En effet ces support sont géralement d'une faible capacité de stockage et sont souvent de type WORM (écriture mais pas effacement).
Toutefois, il existe des solutions qui peuvent correspondre à des petites structures ne necessitant pas forcément de grande capacité de stockage.
Des sytèmes de chargeurs permettent une automatisation, ce qui aura pour effet d'éviter le contact des disques avec l'être humain, négatif pour l'intégrité des données.

\subsection{Disquette}
Les disquettes, très à la mode dans les années 1980, n'est évoqué ici que pour des raisons historiques.
Très limité en terme de capacité de stockage, ce support est aujourd'hui et depuis un certain temps totalement obsolète

\subsection{SSD}
Ce type de support à base de mémoire flash ont pour particularité d'être extrêmement rapide, aussi bien en lecture, qu'en écriture.
Ces éléments ne possèdent pas le défaut de fragilité des disques durs.
En revanche le ratio prix \slash capacité est encore extrèmement élevé, et les capacités sont plus limiteés que les disques durs classiques, ce qui rend leur utilisation comme support pour la sauvegarde ou l'archivage impossible.

\subsection{Services de sauvegardes à distance}
Avec la popularisation des accès internet à haut débit, c'est une méthode en vogue et qui se développe de plus en plus.
Ces services permettent d'effectuer des sauvegardes en ligne, ce qui semble être la solution la plus fiable en terme d'intégrité des données, puisque ces sauvegardes peuvent être stockées n'importe où dans le monde.
En revanche, cela oblige de faire confiance à un tiers puisqu'il aura accès à toutes les données. Même en chiffrant ces données avant de les envoyer, il faudra que ce service soit sûr. En effet les conséquences peuvent être dramatiques si un problème survient chez le tiers alors qu'on doit effectuer une restauration.
De plus, les particuliers ont en général un faible débit montant, ce qui est contraignant dans le cas de sauvegarde sur une grande quantité de données.

%tableau récapitulatif
\begin{table}[h]
 \begin{center}
 \begin{tabular}{|c|c|c|c|c|c|}
  \hline
    Type & Taille & Prix & Intégrité & Vitesse & Simplicité\\
  \hline
    Bandes magnétiques & O & O & O & O & N \\
  \hline
    Disques durs & O & O & N & O & O \\
  \hline
    Supports optiques  & N & O & N & O & N \\
  \hline
    SSD & N & N & O & O & O \\
  \hline
    Sauvegardes à distance & O & N & O & N & N \\
  \hline
 \end{tabular}
 \caption{Tableau récapitulatif sur les dispositifs de stockages}
 \end{center}
\end{table}

\section{Gestion du dépôt}
Il existe aussi plusieurs types de gestion de dépôts. 
En effet il faudra choisir une méthode qui correspond aux besoins de sauvegardes en termes d'accessibilité, de sécurité et de coûts.
Ces méthodes ne sont pas exclusives et pourront être combinées pour arriver à une gestion des sauvegardes et de leurs dépôts plus fine et plus élaborée.

\subsection{En ligne}
Le support de stockage est en permanence connecté. Cela lui donne l'avantage d'être tout le temps accessible et permet d'effectuer une restauration avec le moins de manipulations.
Ce système de stockage peut se mettre en place à l'aide de disques internes, ou comme c'est le cas dans les grandes structures, à l'aide de baies SAN.
En plus d'être coûteuse, la gestion des dépôts en ligne induit des problèmes de sécurité. En effet, les données étant accessibles en permanence, elles peuvent facilement être altérées, de manière intentionnelle, ou suite à une erreur de manipulation.

\subsection{Intermédiaire}
Moins chers, mais aussi moins accessibles que les systèmes en lignes, ces solutions dites intermédiaires semblent un bon compris encore accessibilité, sécurité des données et coûts.
Cela permet aussi de rester assez réactif en cas de restauration.
Un bon exemple serait l'utilisation d'une bibliothèque de bandes magnétiques automatisée, qui tout en offrant une solution déconnectée, permet d'effectuer une restauration rapidement.

\subsection{Déconnecté}
Sauf lorsque les sauvegardes et les restaurations sont réalisées, les sauvegardes sont complètement isolées du point de vu informatique et nécessite une intervention manuelle.
Dans le cas de bandes magnétiques, il s'agira d'une insertion dans un lecteur. Dans le cas d'un disque dur, il s'agira de le connecter à un ordinateur par exemple.
Cette solution est donc orientée sur la sécurité des données et l'accessibilité dépendra de l'emplacement des sauvegardes (typiquement sur site ou hors site) 

\subsection{Externalisation}
Le 11 Septembre l'a montré, le fait d'avoir des sauvegardes n'empêche pas la perte irrémédiable de données.
Pour se prémunir contre des désastres de cette ampleur tel que des feux ou des tremblements de terre, il faudra opter pour un stockage d'une partie ou de toutes les sauvegardes hors site.
Ces endroits où seront stockées ces données peuvent être plus ou moins évolués, d'un simple bureau chez l'administrateur, jusqu'aux chambres fortes.
Dans ce cas, la mise en place d'une restauration peut s'avérer plus longue, et pose donc des problèmes d'accessibilité.

\subsection{Site de sauvegarde}
Si lors d'un évènement particulèrement désastreux, les équipements informatiques sont hors d'état de fonctionner, le fait d'avoir des sauvegardes à jour ne permettra pas une restauration, et l'indisponibilité du système sera alors inévitable.
Dans le cas de cette éventualité, une solution particulièrement couteuse consiste de fournir à un autre site des équipements informatiques nécessaires pour prendre le relais.
Ainsi, en alimentant le second site avec des sauvegardes à jour provenant du premier site, il pourra fonctionner indépendamment et de manière autonome, jusqu'à un rétablissement.

%tableau récapitulatif
\begin{table}[h]
 \begin{center}
 \begin{tabular}{|c|c|c|c|c|c|}
  \hline
    Type & Accessibilité & Coûts & Sécurité & Vitesse de restauration & Simplicité \\
  \hline
    En ligne & O & N & N & O & O \\
  \hline
    Intermédiaire & O & O & O & O & N \\
  \hline
    Déconnecté  & N & O & O & N & N \\
  \hline
    Externalisation & N & N & O & N & N \\
  \hline
    Sites de sauvegardes & O & N & O & O & N \\
  \hline
 \end{tabular}
 \caption{Tableau récapitulatif sur les gestions de dépôt}
 \end{center}
\end{table}

\chapter{Les données}
Lors de la conception de stratégies de sauvegardes, une étape importante consiste à établir quelles seront les données qui sont critiques et qui auront besoin d'être sauvegardées.
Il faudra bien évidemment considérer les fichiers importants pour l'utilisateur, mais aussi ceux qui sont nécessaires au bon fonctionnement de l'ordinateur.
Cette étape est vitale lors de la conception de stratégies de sauvegardes. En effet, choisir de sauvegarder un nombre trop important de fichiers aurait pour conséquences un remplissage de l'espace de stockage et empêcherait l'application de la stratégie de sauvegardes à court terme.
Au contraire, ne pas sauvegarder assez de données aurait, comme on peut s'en douter, des conséquence désastreuses lors d'une perte de données.

\section{Fichiers}
Aujourd'hui, toute donnée informatique est structurée en fichiers.

\subsection{Sauvegarde de fichiers}
C'est évidemment ce qui sera copié dans le cas de sauvegardes classiques.

\subsection{Sauvegarde de parties de fichiers}
Lorsqu'on modifie un fichier, une sauvegarde de type différentielle ou incrémentale aura pour effet d'inclure toute le fichier dans la nouvelle sauvegarde, même si celui-ci n'a été que peu modifié.
Afin de résoudre ce problème de redondance de données, une solution consite à sauvegarder uniquement les blocs qui ont été modifiés plutôt que tout le fichier.
Les procédés de compression et de dé-duplication sont une solution à ce problème.

\section{Systèmes de fichiers}
Il peut être judicieux dans certains cas de sauvegarder des éléments relatifs au système de fichiers.

\subsection{Sauvegarde complète}
Plutôt que de copier l'ensemble des fichiers contenus dans le système, une solution consiste à sauvegarder le système de fichiers dans son ensemble.
Cette technique s'apparente à la réalisation d'images disque.
En copiant tout le système de fichiers à l'aide d'outils de bas niveau (comme dd sous les systèmes UNIX), ce type de sauvegarde est souvent plus rapide, surtout quand les données sont composées de petits fichiers fragmentés.
En revanche, la réalisation de ce type de sauvegardes ne peut pas généralement se faire à chaud, c'est à dire que le système sera indisponible durant toute la durée de la sauvegarde.
De plus, les avantages de ce type de sauvegarde sur un système de fichier peu rempli seront vites effacés car un outil de bas niveau copiera tous les blocs du système de fichiers, sans faire la distinction entre des blocs contenant des données et des blocs non indexés.

\subsection{Identificateurs de changement}
Certains systèmes de fichiers évolués permettent à chaque fichier d'avoir un bit d'archive qui permettra de savoir si des changements ont été opérés sur le fichier depuis la dernière sauvegarde.

\subsection{Versionnage du système de fichiers}
Lorsque elle est mise en place sur un système de fichiers, une version de chaque fichier modifié est automatiquement conservé.
Ainsi, l'utilisateur qui le souhaite peut restaurer un fichier à un instant donné, sans avoir a effectuer aucune manipulation au préalable.

\section{Données en temps réel}
Lorsque la sauvegarde est effectuée à chaud, c'est à dire que la sauvegarde a lieu alors que le système est utilisé normalement, il se peut que la sauvegarde de fichiers ouverts et en cours de modification pose des problèmes.
En effet, comme c'est généralement le cas lors de sauvegardes de bases de données, les données ne reflètent pas leur état à un instant t puisqu'elles ont été modifiées entre le début et la fin de la sauvegarde.
Si des données sont liées, elles peuvent ne plus être cohérentes entre elles, et donc pas exploitables.  

\subsection{Sauvegarde par snapshots}
Disponible sur certains systèmes de stockage, cette fonction permet de faire une copie du système de fichier à un instant donné.
Lors de la prise d'un snapshot, les activités du système sont très brièvement gelées, et on a donc plus de risque d'incohérence des données.

\subsection{Sauvegarde de fichier ouvert}
Certains logiciels permettent la sauvegarde de fichiers ouverts simplement en vérifiant si les fichiers sont utilisés, ou utilisent des verrous.

\subsection{Sauvegarde à froid}
Ce terme est généralement utilisé pour les bases de données.
Lors de ce type de sauvegardes, la base de données devient indisponible le temps de la sauvegarde, et ceci pour éviter toute modification de fichier qui risquerait d'entraîner des incohérences.

\subsection{Sauvegarde à chaud}
Cette fois, ce type de sauvegarde, généralement sur des bases de données, permet d'effectuer des sauvegardes sans pour autant rendre la base indisponible.
Un image des données est d'abord réalisée à chaud, pendant que les modifications sont stockées dans des journaux.
A la fin de la réalisation de l'image initiale, les journaux sont rejoués pour avoir une copie des données conforme à la date de restauration.

\section{Méta-données}
Au-dela des fichiers, il faudra aussi penser à sauvegarder des informations relatives au système.

\subsection{Secteur de boot}
Indispensable pour le démarrage du système, il peut être utile dans certains cas de le sauvegarder.

\subsection{Schémas de partitions}
La aussi, il peut être utile sauvegardeer ces informations qui pourraient être utiles lors d'une restauration rapide du système.

\subsection{Méta-données des fichiers}
Ces méta-données stockent, entre autre, les permissions, les ACL et les propriétaires de chaque fichier. 

\subsection{Méta-données du système}
Ce sont des informations nécéssaires pour le système.

\chapter{Optimisation}
Beaucoup de méthodes d'optimisation pour la sauvegarde et l'archivage sont utilisées, et ont pour rôle d'améliorer les performances de ce processus, comme par exemple les vitesses de sauvegardes, les vitesses de restauration, ou encore l'intégrité et la sécurité des données.

\section{Compression}
Ce mécanisme permet d'influer sur l'espace qu'occuperont les sauvegardes sur l'espace de stockage. 
En effet, il existe de nombreux algorithmes et de procédés plus ou moins efficaces et dépendant du type de données ciblées qui permettront d'alléger les données en taille et donc d'optimiser l'espace disque.

\section{Dé-duplication}
Tandis que la compression n'est ciblée que sur un fichier, voir un groupe de fichiers, la dé-duplication est elle gérée par le support de stockage lui-même.
En analysant les blocs déjà présents sur le support, il optimisera l'espace de stockage en préférant indexer ces blocs plutôt que de les réécrire, ce qui aura pour conséquences une redondance nulle et une optimisation de l'espace disque maximale. 

\section{Duplication}
Les stratégies de sauvegardes ont parfois reccourt à de la duplication afin de manipuler et d'optimiser ces données dans le cas d'une éventuelle restauration, ou tout simplement pour stocker les données dans deux endroits différents dans le but d'améliorer la sécurité.

\section{Chiffrement}
Comme les données sauvegardées sont souvent des données importantes, voir critiques, elles sont souvent confidentielles.
La perte ou le vol de sauvegardes peut alors être lourd de conséquences.
En chiffrant les sauvegardes, cela permettra de sécuriser les données.
Toutefois cela a un coût : Le chiffrage est un processus qui requiert du CPU et qui augmentera le temps des sauvegardes.

\section{Refactoring}
Ce processus consiste a réorganiser les données sauvegardées dans les dépôts de données.
Cela est surtout utile pour les stratégies de sauvegarde basées essentiellement sur des sauvegardes incrémentales.

\section{Staging}
Cela consiste à faire transiter les sauvegardes par un autre support avant que celles-ci n'atteignent l'espace où elles seront stockées.

\chapter{Gestion des processus de sauvegarde}
Il faut comprendre que la sauvegarde est un processus. Les sauvegardes sont indispensable dans tout système informatisé.
Même si les structures sont différentes en terme de grandeur, les objectif et les contraintes restent les mêmes. 

\section{Objectifs}

\subsection{Objectifs de points de restauration}
Cela correspondra au point dans le temps auquel le système sera restauré.
Idéalement, il faudrait avoir une sauvegarde correspondant à la date juste avant la perte de données.
Plus le délai entre chaque sauvegarde sera court, plus on pourra se rapprocher de la date à laquelle les donnée ont été altérées.
Multiplier les sauvegardes a bien évidemment un coût, et il faudra donc ajuster la stratégie en fonction de la structure et des moyens mis à disposition.

\subsection{Objectifs de temps de restauration}
Lors d'un désastre, la durée nécessaire à la restauration sera un élément capital, car il définira le temps pendant lequel le système sera indisponible.

\subsection{Protection des données}
La protection des données est aussi un élément majeur dans les stratégies de sauvegarde.
En effet, il serait paradoxal d'avoir un système ultra-protégé et de laisser les sauvegardes, qui contiennent les même données, accessibles à tous.
Le chiffrement des sauvegarde permettra, entre autre, de protéger ces données sensibles.

\section{Limites}
Pour qu'un schéma de sauvegarde, il faudra prendre en compte les limites de l'environnement.

\subsection{Fenêtre de sauvegarde}
On appelle fenetre de sauvegarde la durée pendant laquelle la sauvegarde est possible, en relation avec l'utilisation plus ou moins importante du système.
Bien évidemment, il faudra prendre en compte cette durée dans la mise en place du schéma de sauvegarde.

\subsection{Performances}
Les sauvegardes sont souvent gourmandes en CPU, et ont donc un impact sur les performances du système.
C'est un élément qu'il faudra aussi prendre en compte dans l'élaboration d'un plan de sauvegarde.

\subsection{Investissements}
Les stratégies de sauvegardes necessitent plus ou moins d'espace disque, ce qui aura un certain coût au final.
Lors du choix de la stratégie de sauvegarde adoptée, l'espace de stockage nécessaire aux sauvegardes devra être calculé, ceci afin de connaître les besoins, et donc les coûts.

\subsection{Limites Réseau}
Les limitations du réseau par lequel transiteront les données dans le cas de systèmes de sauvegarde ditribués sont autant de contraintes à ne pas négliger.

\section{Réalisation}
Concevoir la stratégie de sauvegarde tout en essayant d'atteindre au mieux les objectifs fixés et en prenant en compte les contraintes et les limites établies peut s'avérer fastidieux.

\subsection{Plannification}
L'utilisation d'un plannificateur de tâches plutôt que de dépendre de l'exécution des tâches manuellement apporte un vrai plus et assure que les sauvegardes seront bien réalisées en temps et en heure.
La plupart des logiciels de sauvegardes incluent cette fontionnalité.

\subsection{Authentification}
Afin d'assurer la protection et la confidentialité des données sauvegardées, un processus d'authentification doit être intégré.

\subsection{Chaîne de confiance}
Les opérations sensibles liées à la sauvegarde tel que la manipulation des supports de sauvegardes ne doit être réalisée que par des personnes dignes de confiance et est un élément important dans la sécurisation des données.

\section{Vérification et supervision}
Les sauvegardes sont des données très importantes pour une entreprise. S'assurer que les données sont correctement sauvegardées et avoir des systèmes de vérification, d'alerte et de journaux d'activités est très important.

\subsection{Validation de la sauvegarde}
C'est le processus par lequel les propriétaires des données peuvent avoir des informations sur les sauvegardes concernant leur données.

\subsection{Rapports}
Il est courant qu'un rapport concernant le déroulement de la sauvegarde soit affiché ou envoyé à la personne concernée.
Ce rapport contiendra des informations précieuses sur le processus de sauvegarde et permettra de savoir si la sauvegarde s'est bien déroulée.

\subsection{Journaux}
Un système de journaux d'activités qui permettra la création d'historiques sur les sauvegardes est aussi un élément essentiel pour s'assurer du bon fonctionnement des sauvegardes.

\subsection{Validation}
La validation est une étape tout aussi importante qui consiste à vérifier l'intégrité des données. 
En effet, en effectuant des checksums et hachages sur les fichiers sauvegardés et en les comparant avec les originaux, ont pourra aisément s'assurer de l'intégrité de leur contenu.
De plus, cela permettra de savoir si les fichiers ont été modifiés depuis la dernière sauvegarde et optimisera ces processus en ne copiant que les données nécessaires.

\subsection{Supervision}
Il est possible de superviser les processus de sauvegardes par un centre de supervision.
Ce centre permetttera d'alerter les utilisateurs sur les erreurs qui peuvent se produire sur les sauvegardes automatisées.

\chapter{Loi}

\section{Confusion}
A cause du chevauchement de nombreuses technologies, la confusion est souvent faite entre sauvegardes, archives et systèmes à tolérance de panne.
Les archives sont des données qui sont stockées car elles ne sont plus utilisées, mais qui doivent être conservées pour des raisons légales et historiques.
La sauvegarde consiste en  la copie de données actuellement utilisées, et qui seront conservées dans le cas d'une perte ou altération de ces données.
Quant aux systèmes à tolérance de panne, il n'assurent pas l'intégrité des données en cas de panne, mais assure justement qu'il n'y aura pas de panne.
On peut donc dire que ces trois technologies, même si elles sont proches techniquement, n'ont pas la même finalité et interviennent à des niveaux différents.

\section{Conseils}

\begin{itemize}
\item Plus les données sont importantes, plus le besoin de sauvegarde est important.
\item Les sauvegardes sont seulement utiles si elles sont correctement intégrées dans une stratégie de sauvegarde.
\item Stocker les sauvegardes à côté des données originales n'est pas conseillé.
\item Les stratégies de sauvegardés basées sur l'automatisation et la plannification sont plus sûres qu'une intervention humaine (l'erreur est humaine).
\item Les sauvegardes font intervenir un grand nombre de facteurs externes. C'est donc une opération qui peut échouer pour des raison diverses et variées. Mettre en place un système de supervision est important.
\item Les sauvegardes multiples sur différents médias et stockées a différents endroits est indispensable pour les données les plus critiques.
\item Il est préférable que les sauvegardes soient stockées dans un format libre et standard, surtout pour les archives qui sont suceptibles d'être conservées sur le long terme.
\item S'assurer du bon fonctionnement des processus de sauvegardes est une tâche primordiale et la non-vérification de celles-ci peut être lourde de conséquences.
\item Si cela est possible et envisageable au niveau de l'investissement, il est conseillé de stocker les sauvegardes sur deux supports différents (par exemple bandes magnétiques et disques durs)
\end{itemize}

\section{Evènements}

\begin{itemize}
\item En 1996, un feu au siège social du crédit lyonnais a entraîné la perte de nombreuses données car les bandes magnétiques étaient stockées sur site.
\item Il a été rapporté que 16 évènements de vols de bandes mangétiques appartenant à des organisations majeures ont été recensés au cours des années 2005 et 2006.
\item le 3 Janvier 2008, un serveur de mail appartenant a TeliaSonera, une entreprise de telecoms et fournisseur d'accès nordique, est tombé. Il a été découvert au cours de cet évènement que la dernière sauvegarde exploitable datait du 15 décembre 2007. Trois cent mille comptes mails ont été irrémédiablement affectés. 
\item Le 27 Février 2011, un bug applicatif de GMail a eu pour conséquences la perte de l'ensemble des mails pour 0.02\% des utilisateurs. Une restauration à partir de bandes magnétiques a permis de récupéré l'ensemble des données perdues.
\end{itemize}

\begin{abstract}
\end{abstract}

\end{document}          
